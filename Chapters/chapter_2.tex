% This is an example for a chapter, additional chapter can be added in the 
% skeleton-thesis.
% To generate the final document, run latex, build and quick build commands
% on the skeleton-thesis file not this one.

\chapter{DENSE ARRAY DESIGN FOR OPTIMAL TRANSMISSION USING SURFACE LOOPS}\label{chapters:chapter_2}
\vspace{-7mm}

%% Section
\section{Design Process}\label{sec:ch_2_sec_1}
With the advantages of parallel transmission (pTx) and push for high channel count, it is natural to consider the upper bound of radiating elements in a dense array. 
There has yet to be published works demonstrating the viability of greater than 100 degrees of freedom or elements for static RF shimming, 
in part because this is a convoluted conversation.
Many assumptions must be made before answering anything close to the question of "How many radiating elements does it take to RF shim a region of interest like the brain?"
RF elements come in many shapes and designs, coupling may remove the advantage of densely packed elements if not properly dealt with, 
brain volumes no doubt vary across the patient population, power constraints are unique
to each scanner, and you want the patient to come out of the scanner without a fried brain! This work attempts to address all of these issues for a particular scanner, 
specifically the Philips Achieva 7T with Multix capabilities.
A surface loop is unanimously used across all forms of MR and was chosen as the base element for the dense array.
Surface loops are often used in dense Rx/Tx arrays due to the ease of decoupling neighboring elements by overlapping the loops to cancel their mutual inductance.
A combination of decoupling strategies is used in order pack many elements into an array. Self-Decoupled coils work nicely to decouple neighboring loops in one direction, 
yet coupling 
Power constraints
The upper limit of transmission elements is entirely dependent on the channel count or power available and coupling between neighboring elements. 


%% Subsection
\subsection{Subsection 1}\label{subsec:subsec_2.1.1}


\clearpage

